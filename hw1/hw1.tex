\documentclass[11pt]{article}
\usepackage{enumerate}
\usepackage{url}
\usepackage{listings}
\usepackage{upquote,textcomp}

\setlength{\parindent}{0cm}

\setlength{\parskip}{0.3cm plus4mm minus3mm}

\oddsidemargin = 0.2in
\textwidth = 6.5 in
\textheight = 9.8 in
\headsep = -1in

\lstset{frame=tb,
  language=,
  aboveskip=3mm,
  belowskip=3mm,
  showstringspaces=false,
  columns=flexible,
  keepspaces=true,
  basicstyle={\small\ttfamily},
  numbers=none,
  numberstyle=\tiny\color{black},
  keywordstyle=\color{black},
  commentstyle=\color{black},
  stringstyle=\color{black},
  breaklines=true,
  breakatwhitespace=true,
  tabsize=3
}

\title{Database Systems, CSCI 4380-01 \\
Homework \# 1 \\
Due Monday September 13, 2021 at 11:59:59 PM}
\date{}
\begin{document}
\maketitle

\vspace*{-0.7in}

\noindent{\bf Homework Statement.} This homework is worth 5\% of your
total grade. It has 10 questions with 10 points for each question. You
are required to complete at least 4 queries (equivalent of 2
points). Any points that you did not complete will be added to Midterm
\#1. (For example, if you only solved 4 queries worth 2 points, your
Midterm \#1 will be worth 3\% more).

Remember, practice is extremely important to do well in this class. I
recommend that not only you solve this homework in its entirety, but
also work on homeworks from past semesters. Link to those is already
provided in Teams, which I am repeating here:

\url{http://cs.rpi.edu/~sibel/DBS_Past_Materials/}

This homework aims to test relational algebra first and foremost, and
a bit of normalization theory.

It is recommended that you do the parts in sequence. The questions get
harder and build on your knowledge of relational algebra from previous
parts. Each question is equal weight.

{\bf Database Description.} I don't know about you, but I played quite
a lot of board games during the last few years. So, we will use a
board game database for our homework for this reason. This database is
loosely based on the {\tt boardgamegeek} site. It contains the
following relations:

games(\underline{gameid}, name, year, publisher, min\_players, max\_players, min\_age\_rec, playtime\_min, playtime\_max, iscooperative, description, link) \\
gametypes(\underline{gameid, gametype}) \\
gamecategories(\underline{gameid, category}) \\
gamemechanics(\underline{gameid, mechanic})  \\
gamedesigners(\underline{gameid, designername}) \\

onlinegamesites(\underline{siteid}, url, price\_per\_month, notes) \\
gamesonsite(\underline{siteid, gameid}, isfree, min\_players, max\_players) \\

gameprices(\underline{gameid, storename}, price) \\
gamereviews(\underline{gameid, userid}, review\_text, review\_date, stars, num\_likes) \\
awardsnominations(\underline{gameid, awardname}, year, iswinner) \\

We store information about games as well as types, categories,
mechanics and designs for each game. There are a number of online
sites where a game can be played online (e.g. boardgamearena), for
each site we have subscription prices as well as which game can be
played on which site. In an online site, min and max player
restrictions for the game may be different than a physical board
game. Finally, games may be nominated for awards and if they win the
award, {\tt iswinner} is set to true.

Additionally, we store where the game is sold and at which price in
{\tt gameprices}.

Finally, in our game community, we have a number of user reviews and
each review is by a user at the site. The reviews have text, star
rating and number of likes by other member of the same community.

Note: All date fields are formatted as \verb+mon-day-year+,
e.g. \verb+01-31-2020+. You can assume that you can check if a date
value {\tt X} comes after another value {\tt Y} by checking whether
{\tt X $>$ Y}. 

\newpage

{\bf Question 1.}
Write the following queries using relational algebra. You may use any
valid relational algebra expression, break into multiple steps as
needed. However, please make sure that your answers are well-formatted
and are easily readable. Also, pay attention to the attributes
required in the output!


\begin{enumerate}  [(a)]
\item Return {\tt name}, min and max {\tt playtime} and {\tt link} of
  all games that can be played with 4 people, came out in 2020
  and were published by \verb+'Rio Grande Games'+.
  %% select only

\item Return the {\tt name} and {\tt designername} of games that won
  or were nominated for the \\
  \verb+'Golden Geek Most Innovative Board Game'+ award in 2019 or
  2020.
  %% select and join

\item Return the {\tt userid} of all users who never reviewed a game
  with the \verb+'Loose a Turn'+ mechanic.
  %% set subtraction
  
\item Return {\tt gameid} of all award winner games in
  categories \verb+'Exploration'+ and
  \verb+'Adventure'+ that do not involve any \verb+'Dice Rolling'+
  mechanic.
  %% select/project and set operations

\item Return the {\tt name, publisher} of all names in the
  \verb+Strategy+ category that won an \verb+'SXSW'+ award and are
  either available for less than \$40 in a store or can be played
  online.
  %% select/project/join and (union or join)

\item Find the {\tt name, publisher} of {\tt cooperative} games in
  the \verb+'Farming'+ category that are either of type
  \verb+'Strategy'+ or have the \verb+'Hidden Victory Points'+
  mechanic.
  %% select/join and set operations

\item Return the {\tt storename} of stores with the cheapest price for
  the game named \verb+'Beyond The Sun'+ that came out in \verb+2019+.

  Assume there is a single game with this name. It is possible that
  multiple stores have the same cheapest price, if so, all such stores
  must be returned.
  
  %% theta join
  

\end{enumerate}

{\bf Question 2.} For the following relations, (a) find and list the
keys, (b) check whether they satisfy BCNF, discuss why or why not, (c)
check whether they satisfy 3NF, discuss why or why not.

To show that a relation is not in BCNF or 3NF, you only need to show a
violation. To show that they are in BCNF or 3NF, check each functional
dependency and discuss why it is ok.

\begin{enumerate}

\item $R1(A,B,C,D,E,F,G)$, ${\cal F} = \{ABC\rightarrow DG, G\rightarrow AEF \}$

\item $R2(A,B,C,D,E,F,G)$, ${\cal F} = \{ABD\rightarrow CEFG, AE\rightarrow BCDG \}$

\item $R3(A,B,C,D,E,F,G)$, ${\cal F} = \{AB\rightarrow CDE, BE\rightarrow F, F\rightarrow G \}$

\end{enumerate}


{\bf SUBMISSION INSTRUCTIONS.} Submit a single PDF or Text
document for this homework using Submitty. No other format and no hand
written homeworks please. No late submissions will be allowed.

If the Submitty for homework submissions is not immediately available,
we will announce when it becomes available on Submitty. 

{\bf Help with relational algebra formatting.} While in class I have
been using a text version of relational algebra, which I have allowed
for many years for students who do not want to figure out the Greek
symbols. However, many past solutions use the more standard version
with Greek symbols. You can use either one in your solutions, but do
not mix and match. Use one consistently.

I present you with the full syntax here in both ways (as well as the
Latex symbols for it). Note that for the standard version, I simply
use the Math mode in Latex.

\begin{tabular}{l|l|l}
  Operation & Text Version & Standard Version \\ \hline
   Set Union & R union S & $R \,\cup\, S $ \\
   Set Intersection & R intersect S & $R \,\cap\, S $ \\
   Set Difference & R - S & $R \,-\, S $ \\
   Rename &   T(A,B,C) = R & $\rho_{T(A,B,C)} (R)$ \\
   Select & select\_\{C\} (R) & $\sigma_{C} (R)$ \\
   Project & project\_\{A1,..,An\} (R) & $\pi_{A1,..,An} (R)$ \\
   Cartesian product & R x S & $R \,\times\, S$ \\
   Theta-Join & R join\_\{C\} S & $R \,\bowtie_{C}\, S$ \\
   Natural Join & R * S & $R * S$ \\
  \end{tabular}

As an additional help, I format one of the queries we did in class in
the standard format below for two equivalent solutions. Please format
to make sure your queries are readable by a human.

Query: Find the id and title of all movies starring a hero with the
power to \verb+'Stop Time'+.

Solution 1 (text format)

T1 = project\_\{heroname,multiverseid\} (select\_\{power = 'stop time'\}(MarvelHeroes)) \\
T2 = project\_\{heroname,multiverseid\} ( select\_\{power = 'stop time'\}(DCHeroes)) \\\
T3(heroname1,multiverseid1) = T1 union T2 \\
T4 = T3 join\_\{heroname=heroname1 and multiverseid=multiverseid1\} HeroInMovie \\
T5(heroname1,movieid1, heroname, multiverseid1)  = T4 \\
T6 = project\_\{movieid, title\} (T5 join\_\{movieid = movieid1\} Movies) \\


Solution 2 (standard format)

\begin{eqnarray*}
& & T1 = \pi_{heroname,multiverseid} (\sigma_{power = 'stop time'}(MarvelHeroes) \\
& & T2 = \pi_{heroname,multiverseid} (\sigma_{power = 'stop time'}(DCHeroes) \\
& & T3(heroname1,multiverseid1) = T1 \cup T2 \\
& & T4 = T3 \bowtie_{heroname=heroname1 and multiverseid=multiverseid1} HeroInMovie \\
& & T5(heroname1,movieid1, heroname, multiverseid1)  = T4 \\
& & T6 = \pi_{movieid, title} (T5 \bowtie_{movieid = movieid1} Movies)
\end{eqnarray*}


\end{document}

